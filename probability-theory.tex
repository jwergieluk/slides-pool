\section{Measure theory}


\begin{frame}
    {Caratheodory extension theorem}
        
    \begin{theorem}
        Let $\mu: R \to \bR_{\geq 0}$ be a pre-measure on a ring $R$. Then $\mu$
        can be extended to a measure $\tilde\mu$ on the $\sigma$-algebra $\sigma(R)$. 
    \end{theorem}

    \begin{enumerate}
        \item A ring $R$ contains $\emptyset$, is closed under finite
            intersections and finite differences, i.e.\ $A,B\in R \impl A\cap
            B\in R, A\setminus B\in R$. 
        \item Premeasure $\mu$ is defined as a positive, $\sigma$-additive set
            function in a ring, with the property $\mu(\emptyset)=0$.
    \end{enumerate}

    Kallenberg has a version of this theorem for semi-rings. 
\end{frame}


\begin{frame}
    {Topologies on product spaces}
    
    \begin{enumerate}
        \item Product topology. 
        \item Box topology.
    \end{enumerate}


\end{frame}

\begin{frame}
    {Cylinder sets}
    
    \begin{enumerate}
        \item Cylinder sets are clopen.
    \end{enumerate}


\end{frame}


\subsection{Monotone Class Theorem}

\begin{frame}
    {Set systems: $\pi$-systems, $\lambda$-systems, $\sigma$-algebras}
    
    The following collections of subsets of some space $\Omega$ are of interest.

    \begin{enumerate}
        \item $\cC$ is called $\pi$-system, if $A,B\in \cC \impl A \cap B \in
            \cC$, i.e., $\cC$ is closed under finite intersections.
        \item $\cD$ is called $\lambda$-system (or Dynkin-system), if 
            \begin{enumerate}
                \item $\Omega\in\cD$,
                \item $A,B\in\cD$ and $B \subset A \impl A \setminus B \in\cD$,
                \item  and $A_1,A_2,\ldots \in\cD$ and $A_n \uparrow A \impl A
                    \in \cD$.
            \end{enumerate}
        \item $\cA$ is called $\sigma$-algebra, if 
            \begin{enumerate}
                \item $A\in\cA \impl A^{c}\in\cA$, 
                \item $A_1,A_2, \cdots \in\cA \impl \cup_{n\geq 1} A_n \in\cA$, and
                \item $A_1,A_2, \cdots \in\cA \impl \cap_{n\geq 1} A_n \in\cA$.
            \end{enumerate}
    \end{enumerate}

    \begin{theorem}
        $\cA$ is a $\sigma$-algebra iff $\cA$ is both a $\pi$-system and a
        $\lambda$-system.
    \end{theorem}
\end{frame}


\begin{frame}
    {Monotone class theorem}
    
    \begin{theorem}[Monotone class theorem]
        Let $\cC$ be a $\pi$-system contained in a $\lambda$-system $\cD$. Then
        $\sigma(\cC) \subset \cD$.
    \end{theorem}

    A corollary: If $\cC$ is a $\pi$-system, then $\sigma(\cC)=\lambda(\cC)$. 

\end{frame}


\begin{frame}
    
    \begin{proof}[Monotone Class Theorem]
        \begin{enumerate}
            \item It suffices to show that $\lambda(\cC)$ is a $\pi$-system, which implies that 
                $\lambda(\cC)$ is a $\sigma$-algebra.

            \item $\cD_2 = \left\{ B \in\lambda(\cC) : B \cap A \in
                \lambda(\cC) \ \forall A \in \lambda (\cC) \right\}$ is the
                largest subset of $\lambda(C)$ closed under finite
                intersections. We show $\cD_2 = \lambda(\cC)$. 

            \item $\cD_1 = \left\{ B \in\lambda(\cC) : B \cap A \in
                \lambda(\cC) \ \forall A \in \cC \right\}$ is a
                $\lambda$-system and $\cC \subset \cD_1$.  Therefore $\cD_1 =
                \lambda(\cC)$.

            \item We get $\cD_2 = \left\{ B \in\lambda(\cC) : B \cap A \in
                \cD_1 \ \forall A \in \cD_1 \right\}$ is a $\lambda$-system and is
                closed under finite intersections. Moreover $\cD_2 = \lambda(\cC)$.
                So $\lambda(\cC)$ is closed under finite intersections, which
                implies $\lambda(\cC) = \sigma(\cC)$.
        \end{enumerate}
    \end{proof}
\end{frame}


\begin{frame}
    {Functional monotone class theorem}
    
    \begin{theorem}
        Let $\cC$ be a $\pi$-system on $\Omega$, and $\cH$ a class of real-valued
        functions on $\Omega$ satisfying:
        \begin{enumerate}
            \item $\cH$ is a vector space.
            \item $1\in\cH$.
            \item $1_{A}\in\cH$ for all $A\in\cC$.
            \item If $\left( f_n \right) \subset \cH$ is an increasing sequence of non-negative
                functions with $f = \sup_{n} f_n$, then $f\in\cH$.
        \end{enumerate}
        Under the above conditions $\cH$ contains all real-valued bounded $\sigma(\cC)$
        measurable functions on $\Omega$.
    \end{theorem}

    \begin{proof}
        TODO
    \end{proof}
\end{frame}


\begin{frame}
    {Families of measurable functions}
        
    Let $(E_i, \cE_i)_{i\in I}$ be a family of measurable spaces, and $\cC_i$
    are $\pi$-systems generating $\sigma(\cC_i)=\cE_i$. Moreover, let $f_i :
    \Omega \to \E_i$ be a family of $\cE_i$-measurable functions.  We define
    \begin{equation*}
        \cC = \left\{  \bigcap_{j\in J} f_j^{-1}(A_j) \,:\,  A_j \in \cC_j, \ J \subset I \text{ finite} \right\}.
    \end{equation*}
    Then
    \begin{enumerate}
        \item $\cC$ is a $\pi$-system, and
        \item $\sigma(\cC) = \sigma\left( f_i : i\in I \right)$.
    \end{enumerate}

$\sigma\left( f_i : i\in I \right)$ is the smallest $\sigma$-algebra on
$\Omega$ rendering all $f_i$ measurable, and is generated by the sets $\left\{
f_i^{-1}(A_i) : A_i \in \cE_i \right\}$.
\end{frame}


\begin{frame}
    {A version of the Monotone Class Theorem}

    We specialise the MCT to the case of the $\pi$-system defined above.
    \begin{theorem}
        Let $\cH$ be a real vector space of functions, such that
        \begin{enumerate}
            \item $1\in\cH$. 
            \item $\cH$ contains all functions of the form
                \begin{equation*}
                    1_{\bigcap_{j\in J} f_j^{-1}(A_j)} (x) = 
                    \prod_{j\in J} 1_{A_j} \left( f_{j}(x) \right), \ A_j \in \cC_j, J \subset I \text{ finite}.
                \end{equation*}
            \item If $\left( f_n \right) \subset \cH$ is an increasing sequence of non-negative
                functions with $f = \sup_{n} f_n$, then $f\in\cH$.
        \end{enumerate}
    \end{theorem}

\end{frame}


\begin{frame}
    {Uniqueness of measures}

    Fix a measurable space $\left( \Omega, \cE \right)$. 

    \begin{theorem}[Uniqueness of finite measures]
        Let $\mu$ and $\nu$ be finite measures on $(\Omega, \cE)$ agreeing 
        on a $\pi$-system $\cC$ which generates $\sigma(\cC)=\cE$. If $\mu(\Omega)=\nu(\Omega)$, then
        $\mu=\nu$.
    \end{theorem}
    \begin{proof}
        $\cD = \left\{ A \in \cE : \mu(A)=\nu(A) \right\}$
        is a $\lambda$-system containing $\cC$, and therefore $\cD=\sigma(\cC)=\cE$.
    \end{proof}

    A version for $\sigma$-finite measures.

\end{frame}



\begin{frame}
    {$\sigma$-algebras and functional dependence}
    
    \begin{theorem}
        Let $f: \Omega \to E$ be a $\cE$-measurable function. 
        \begin{enumerate}
            \item The set  
                \begin{equation*}
                    \cH = \left\{ h \circ f \,:\, h:\Omega\to\R \text{ bounded and measurable} \right\}
                \end{equation*}
                contains all real-valued $\sigma(f)$-measurable functions. 
            \item For a $\phi: \Omega\to \bar\R$ there exist an $h\in\cE$ such that
                \begin{equation*}
                    \phi = h\circ f.
                \end{equation*}
        \end{enumerate}
    \end{theorem}

    \begin{proof}
        $\cH$ satisfies the conditions of the functional version of MCT. 
    \end{proof}

\end{frame}


\begin{frame}
    {How does $\phi \in \sigma(f_i : i\in I)$ look like?}
    
    Essentially, $\phi$ is a function of at most \emph{countably} many $f_i$. 

    \begin{theorem}
        Let $\left\{ \cE_i : i\in I \right\}$ be a family of $\sigma$-algebras
        on $\Omega$. Then for every $A \in \cE = \sigma( \cE_i : i\in I )$
        there exists a countable set $J \subset I$ such that $A \in \sigma(
        \cE_j : j \in J)$. 
    \end{theorem}

    In functional form this reads:
    \begin{theorem}
        Let $\left\{ f_i: \Omega\to\cE_i \right\}_{i\in I}$ be a family of
        functions with $f_i\in\cE_i$, and $\phi\in \cE =\sigma\left( f_i : i\in
        I \right)$ a numeric function on $\Omega$. Then there exists a
        countable set $J \subset I$ such that 
        \begin{enumerate}
            \item $\phi \in \sigma\left( f_j : j\in J \right)$.

            \item There is a $\bigotimes_{j\in J} \cE_j$-measurable function
                $h: \prod_{j\in J} E_j \to \bar\R$ with $\phi = h \circ \left(
                f_j \right)_{j\in J}$.
        \end{enumerate}
    \end{theorem}
\end{frame}

\begin{frame}
    
    \begin{proof}
        TODO
    \end{proof}

\end{frame}







\section{Probability theory}

\begin{frame}
    {Density transformation formulas}
    
    \begin{block}{One-dimensional version}
        \begin{equation*}
            p_{h(X)} (y) = \frac{p_X (h^{-1}(y))}{| h'(h^{-1}(y)) |} 1_B(h^{-1}(y))
        \end{equation*}
        
    \end{block}

    \begin{block}{Jacobi's Transformation Formula}
        Let $X=(X_1, \cdots,X_n)$ have joint density $f$ and $g:\R^n \to \R^n$ be
        continuously differentiable and injective, with non-vanishing Jacobian
        $J_g(x)_{ij}= \frac{\partial g_i}{\partial x_j} (x)$.
        Then $Y=g(X)$ has density 
        \begin{equation*}
        f_Y(y) = f_X(g^{-1}(y)) \ |\det J_{g^{-1}}(y) | \ 1_{g(\R^n)}(y).
        \end{equation*}
    \end{block}

    \emph{Sources: } Jacod-Protter, Chapter 12
\end{frame}




\begin{frame}
    {Workhorses of Probability Theory}
    
    \begin{block}{Monotone convergence theorem}
        If $X_n$ are positive and increasing a.s.\ to $X$, then 
        $\lim_{t\to \infty} E \left\{ X_n \right\} = E \left\{ X \right\}$.
        ($E\left\{ X \right\} =\infty$ allowed)
    \end{block}

    \begin{block}{Fatou's lemma}
        If $X_n$ satisfy $X_n \geq Y$ a.s.\ for some integrable $Y$, for all $n$,
        then 
        $E\left\{ \lim\inf_{n\to \infty} X_n \right\} 
                \leq \lim\inf_{n\to \infty} E\left\{ X_n \right\}$.
    \end{block}
    
    \begin{block}{Lebesgue's dominated convergence theorem}
        If $X_n$ converge a.s.\ to $X$ and if $ |X_n| \geq Y$ a.s. for some integrable $Y$,
        for all $n$, then $X_n$ and $X$ are integrable and 
        $\lim_{n\to \infty} E\left\{ X_n \right\} = E\left\{ X \right\}$.
    \end{block}
\end{frame}






