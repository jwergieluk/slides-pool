
%% An introduction to one period models 


\begin{frame}
    {One-period market models}
  
    Fix a probability space $\left( \Omega, \cF, P \right)$ and consider a
    market model consisting of $d+1$ assets at time points $t=0$ and $t=1$.  

    The asset prices at $t=0$ are known real numbers 
    $$\bar{\pi} = \left( \pi^{0}, \cdots, \pi^{d} \right) \in \R_{\geq 0}^{d+1}$$
    and the prices at time $t=1$ are real-valued random variables
    $$\bar{S} = \left( S^0, \cdots, S^{d} \right).$$

    The zeroth asset has a special role, i.e. it is a \emph{risk-free asset} such
    that $\pi^{0} =1$ and $S^=1+r$, $r>-1$. The number $r$ is called the 
    risk-free rate.
\end{frame}


\begin{frame}
    {Portfolios}
    
    An investment, at time $t=0$ can be specified via a portfolio 
    $\xi \in \R^{d+1}$, where $\xi^{i}$ represents the number of 
    bought shares of $i$-th asset.

    The price of a portfolio at time $t=0$ is given by 
    $\bar{\xi} \cdot{} \bar{\pi}$.
\end{frame}


\begin{frame}
    {Arbitrage opportunities}
    
    An \emph{arbitrage opportunity} is a portfolio $\bar \xi \in \R^{d+1}$ with
    \begin{enumerate}
        \item $\bar \xi \ \bar \pi \leq 0$.
        \item $\bar{\xi} \ \bar{S}  \geq 0$ $P$-a.s.
        \item $P \left(  \bar{\xi} \ \bar{S} > 0   \right) >0$.
    \end{enumerate}
   
    A market model admitting no arbitrage opportunities is called
    \emph{arbitrage-free}. 

    \begin{block}{Remarks.}
        \begin{enumerate}
            \item The measure $P$ enters the definition of the arbitrage
                opportunity only through the null-sets of $P$, and thus $P$ may
                be replaced by any $Q \approx P$.
            \item In the above difinition, $\bar{\xi} \ \bar{\pi}\leq 0$
                may be replaced by $\bar{\xi} \ \bar{\pi} =0$.
        \end{enumerate}
    \end{block}
\end{frame}


\begin{frame}
    {Risk-neutral measures and FTAP}
    
    A probability measure $P^{*}$ is called a \emph{risk-neutral measure}
    or \emph{martingale measure} if the discounted prices are martingales, i.e.
    \begin{equation}
        \pi^{i} = E^{*} \left[ \frac{S^{i}}{1+r} \right], \ 
        i \in \left\{ 1,\cdots,d \right\}.
    \end{equation}

    Let $\cP$ be the set of all equivalent martingale measures, i.e.\ 
    \begin{equation}
        \cP = \left\{ P^{*} \vb P^{*} \approx P, 
        P^{*} \text{ is a martingale measure}  \right\}.
    \end{equation}

    \begin{theorem}[The Fundamendal Theorem of Asset Pricing]
        A market model is arbitrage-free iff $\cP \neq \emptyset$. 

        In this case the Radon-Nikodym density $\frac{d P^*}{d P}$ is bounded. 
    \end{theorem}
\end{frame}


\begin{frame}
    {More on risk-neutral measures}
        
    We introduce the $i$-th \emph{discounted net gain} 
    \begin{equation}
        Y^i = \frac{S^i}{1+r} - \pi^{i}.
    \end{equation}

    If $P^{*} \in \cP$ then $E^* \left[ Y^i \right] =0$.

\end{frame}


\begin{frame}
    {Attainable payoffs and the Law of One Price} 
    
    The finite-dimensional vector space $\cV = \left\{ \bar{\xi} \ \bar{S} \vb
    \bar{\xi} \in R^{d+1} \right\}$ is called the space of \emph{attainable payoffs}.
    The operator $\pi (V) = E^{*} \left[ \frac{V}{1+r} \right]$ associated
    with a measure $P^* \in \cP$ is linear and can be extended to the space $L^1
    (P^*)$ of $P^*$-integrable random variables. This extension is not unique
    in general. 

    \begin{theorem}[The Law of One Price]
        In an arbitrage-free market let $V\in \cV$ and
        \begin{equation}
            V = \bar{\xi} \ \bar{S} = \bar{\zeta} \ \bar{S} \quad P-\text{a.s.}
        \end{equation}
        Then $\bar{\xi} \ \bar{\pi} = \bar{\zeta} \ \bar{\pi}$.
    \end{theorem}

    Interpretation: The portfolios having the same payoffs must also the 
    same price. 
\end{frame}





